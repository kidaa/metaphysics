\documentclass[12pt]{report}
\usepackage{hyperref}   % use for hypertext links, including those to external documents and URLs
\usepackage{makeidx} 
\makeindex

\begin{document}

\title{Reality in 40 Verses}
\author{Ramana Maharishi}
\maketitle


\tableofcontents

\chapter{Introduction}

\chapter{Invocation}

\section*{i}
\section*{ii}

\chapter{Text}


\section*{1}

\begin{quote}
Because the world is seen, we have to infer a common (a Lord)
possessing unlimited powers to appear as the diversity. The pictures
consisting of names and forms, the seer, the canvas, the light ----
all these are He Himself.
\end{quote}

\index{world}\index{Lord}\index{diversity}\index{Self}

The Forty begins here. To understand Bhagavan's meaning we have to use
the key with which he supplies us in the Invocation. There he declares
Reality to be the thought-free Awareness which dwells in the
heart. Here he brings in the world in order to meet on their own
ground those disciples who do perceive a ``real'' external world. He
is saying something like this: ``You see a world and ascribe an
omnipotent creator to it. But as we have already seen, this creation
is only an appearance, a manifestation of that Awareness of which we
were speaking. It has no more reality in itself than have the pictures
projected on a screen.'' From the heart thoughts spontaneously rise,
like vapour from the ocean, and turn into a kaleidoscopic world of
names, forms, colours, sounds, smells and other impressions. These are
in it, or on it as on a canvas of which the heart is itself the seer
and the sight.

Pure Consciousness or Pure Mind is thus the pictures, the screen, the
seer, and the light or sight.

\section*{2}

\begin{quote}
All schools of thought postulate the fundamental triad --- God, soul
and world --- although all three are manifestations of the One. The
belief that the three remain eternally lasts only as long as the ``I''
or ego lasts. To destroy the ego and remain in one's own state is best.
\end{quote}

\index{God}\index{soul}\index{world}\index{Self}

Most religions are based on the assumption that the triad mentioned in
the text is eternal. Bhagavan rejects this assumption as being the
child of the ignorant ego which mistakes itself for the body. The
``I-am-the-body'' notion compels the admissions of an individuality
(jiva), a world, its creator, as three distinct, perennial,
co-existing entities. Bhagavan, as we have seen, perceives a single
existence of which these three are an illusory manifestation which,
however, vanishes the moment the eternal ``I'' is apprehended and the
ego perishes.

\section*{3}

\begin{quote}
Of what avail to debate whether the world is real or unreal, sentient
or insentient, pleasant or unpleasant? Extingishing the ego,
transcending the world, realising the Self --- that is the state which
is dear to all, and free from the sense of unity and duality.
\end{quote}

\index{philosophy}

The same line of thought continues. Destruction of the ego is a
\emph{sine qua non} for the realisation of the Self within the
heart. It brings to an end all speculation about reality and
unreality, God and world, whose true nature will be revealed in actual
experience. This is the most blissful attainable state and beyond the
plurality of the illusory world.

\section*{4}

\begin{quote}
If the Self be with form, God and the world will be also. If one be
formless oneself, how and by whom can their forms be seen? Can their
be sight without eyes? The Self is the eye, the limitless Eye.
\end{quote}

\index{form}\index{formless}

This refers to the \emph{jnani}, who although having a body seem
himself as bodiless and formless, and so cannot see God, or in fact
see anything with form. The \emph{ajnani} (the non-realised),
perceiving himself as a body, takes God also to be a body and worships
him in all sorts of material, formal representations. Yet the fact
remains that even he perceives everything through his own formless
Self, which we have granted to be the only seers, the only knowledge
there is --- the ``limitless Eye''. Those who condemn idol-worship
forget that they themselves worship material symbols and icons, and
attribute to God forms, dimensions, positions, even sentiments and
sense-perceptions exactly as they do to themselves. Having no
experience or conception of a formless omniscient spirit, they feel
literally lost lost at the idea of worshipping something not
represented in a form. God, thus, appears according to the degree of
realisation of one's Self.

``Can there be sight without eyes?'' means that without consciousness
there can be no knowledge of anything, just as without a lamp none of
the objects present in a darm room can be seen. Can there be a world
to an unconscious man?

\section*{5}

\begin{quote}
The body is in the form of, and includes, the 5 sheathes. Is there a
world apart from the body? Has anyone without a body seen the world?
\end{quote}

\index{body}\index{sheathes}\index{world}\index{ghost}

The body is a complex structure containing a large number of
instruments or organs which the Self, as ego uses for a large number
of purposes, including among others those of hearing, smelling,
seeing, thinking, feeling, memorizing and reasoning. The materials out
of which these instruments or parts are made vary from the grossest to
the finest. The emph{Shastras} (scriptures) have arranged them in five
groups. To each group one sheath of \emph{kosha} is assigned. The
\emph{kosha} dealing with purely physical matter is called
\emph{annamayakosha} (the sheat of food). The \emph{pranamayakosha}
(the vital sheath) looks after the fivefold functions of the vital
energies --- breathing, assimilatin, generation, excretion and
locomotion. The \emph{manomayakosha} (mental sheath) contains the
faculties of mentation. The \emph{vijnanamayakosha} is the sheath of
the intellectual and reasoning faculties, of scientific and
philosophic thinking, and last is the \emph{anandamayakosha}, the
sheath of bliss, or causal sheath, which stores up within itslef the
karmic seeds of every birth and is concerned with that state in which
profound peace is enjoyed by the dreamless sleeper. This \emph{kosha}
is made of the finest substance, \emph{sattva}, which in itself is
happy, due to its freedom from grossness and its close proximity to
the blissful Self.

Thus the term body includes all these \emph{koshas}, whose appearance
and disappearance cause the appearance and disappearance of all
objective and subjective perceptions. Assumption of a body is
therefore necessary for the world's enjoyment and the body owes its
existence, as we shall see in the next verse, to the five senses,
which are the properties of the mind.

\section*{6}

\begin{quote}
The world is but the fivefold sense-objects, which are the results of
the five senses. Since the mind perceives the world through the
senses, is there a world without the mind?
\end{quote}

\index{world}\index{senses}\index{mind}

Through the sensory organs lodges in the five \emph{koshas}, the
senses display before the mind a variety of objects --- physical,
vital, emotional, mental and intellectual. Apart from the five sense
perceptions, there all sorts of other internal senses which also arise
from the mind, work through the mind, and are understood by the mind
--- such as the senses of time, of space, or ``I'' and ``mine'', and
the artistic, ethical, religious and spiritual senses for
instance. Since all these senses form the world we known and have one 
common origin, which is the mind, the world cannot therefore be other
than that mind.

\section*{7}

\begin{quote}
Although the world and the awareness of it rise and set together, it
is by awareness that the world is known. The source from which they
both rise, and into which they both set, always shines without itself
rising or setting. That alone is real.
\end{quote}

\index{world}\index{awareness}

This verse is reminiscent of the Invocation and confirms the previous
verse, which make awareness the criterion of existence as well as the
source of the world. Awareness ``always shines'' as the ``limitless
Eye'' mentioned in verse four, the eternal Knower. It goes without
saying that the appearance of the world is simultaneous with the
awareness of it, and disappearance of the world simultaneous with the
withdrawal of that awareness. For the fact of the awareness of the
world is the fact of its existence. We cannot affirm the existence of
an object without first affirming awareness of it. Therefore awareness
is the only Reality there is.

\section*{8}

\begin{quote}
In whatever name and form the nameless and formless is worshipped,
therein lies the path of its realisation. Realising one's truth as the
truth of that reality, and merging into it, is true realisation.
\end{quote}

\index{form}

All roads lead to Rome --- all sincere worship comes from the heart,
and leads to the formless God in the heart. To believe that one's
reality is the same as God's is an important step towards the
realisation of Him as Pure Consciousness and the process of merging
into Him. How many millions of innocent human beings would have been
spared the horror of religious persecutions throughout the centuries
in the name of God, and how many wars would have been prevented, had
this truth been accepted as the one truth underlying all religions,
the basic world faith!

\section*{9}

\begin{quote}
The dyads and triads rest on the basic One. Inquiring about that One
in the mind, they will disappear. Those who see thus are the seers of
truth: they remain unruffled.
\end{quote}

\index{dyad}\index{triad}

The dyads are the pairs of opposites --- knowledge and ignorance,
light and darkness, happiness and miserty, birth and death, etc. The
triad is triple principle of seen, seer and sight; object, subject and
the perception of the former by the latter. As all the numbers stand
on, and originate from, the first number, so are the dyads and triads
based on, arising from, and ofthe same nature as the one seer, the
perceiving mind. He who realises the world as such retains a uniform
serenity in all conditions of life.


\section*{10}

\begin{quote}
Knowledge and ignorance are interrelated: the one does not exist
without the other. Inquiring to whom is it that knowledge and that
ignorance, and arriving at their root cause, the Self, this is true
knowledge. 
\end{quote}

\index{knowledge}\index{ignorance}\index{inquiry}\index{Self}

To speak of ignorance is to admit its opposite --- knowledge --- and
\emph{vice versa}. Until we become aware of an object we remain
ignorant of its existence. To learn a lesson is to admit our previous
ignorance of its content. Knowledge is thus the light which clears
away the darkness of ignorance. But knowledge and ignorance which
pertain to external objects are mere modes of thought. They come and
go, and are therefore of no consequence in the search for Truth. What
is of consequence is their knower, who is fixed, changeless, also
called first principle because he is efficient, causeless, the eternal
thinker, who precedes and survives all his thoughts --- ``the basic
One'' (verse nine).

\section*{11}

\begin{quote}
Is it not ignorance to know all but the all-knowing Self? When the
latter, the substratum of both knowledge and ignorance, is known,
knowledge and ignorance themselves both disappear.
\end{quote}

\index{knowledge}\index{ignorance}

It is of course foolish to know about everything in the world, and
remain ignorant of one's own Self. Knowledge of the perishable -- the
universe and its contents -- perishes with the body, and cannot be
transferred to another body, except perhaps as tendencies or abilities
in the perishable too, which may not have any spiritual value in a
future life. The imperishable alone endures and gives imperishable
satisfaction, and this lies wholly within ourselves, who are the
source and ground of both knowledge and ignorance -- that is, of all
experiences whatever.

\section*{12}

\begin{quote}
True knowledge is neither knowledge or ignorance. Objective knowledge
is not true knowledge. Because the Self is self-effulgent, having no
second to know or be known, it is Supreme Knowledge -- not empty
nothingness.
\end{quote}

\index{knowledge!objective}\index{void}

This continues the theme of verses ten and eleven. We have seen tht
objective knowledge is knowledge of the perishable, the apparent, the
non-existent, the unreal (see Invocation). Self-awareness is true
knowledge, because it is absolute, i.e. changeless, non-dual,
ever-pure (thought-free). This purity is not emptiness because of the
lack of perceivable objectives in it, but the ever-shining plenum of
Awareness-Being (\emph{chit-Sat}).

\section*{13}

\begin{quote}
The Self alone is knowledge, is truth. Knowledge of the diversity is
ignorance, is false knowledge. Yet ignorance is not apart from the
Self, which is knowledge. Are the ornaments different from the gold
which is real?
\end{quote}

\index{diversity}

So the world with all its multiplicity of shapes, colors, smells
tastes and so forth is nothing but pure consciousness in substance,
like variously-shaped jewelry which is nothing but gold. To perceive
shapes, colors, smells and the like as different from one another is
ignorance, is illusion, but to see them as the single substance out of
which they are made -- the pure mind -- is true knowledge.

``Yet ignorance is not apart from the Self'' because all experiences
as thoughts come from the Self and are witnessed by it (verses six and
seven). 

\section*{14}

\begin{quote}
The ``I'' existing, ``you'' and ``he'' also exist. If by investigating
the truth of the ``I'' the ``I'' ceases, ``you'' and ``he'' will also
cease and will shine as the One. This is the natural state of one's
being.

\end{quote}

``You'' and ``he'' are the world; it stands and falls with the ``I''
or ego, which constructs it. Realising one's being is realising the
whole world to be the same effulgent being -- ``the One''. This state
of being is experienced by the Self-realised man in the waking state
consciously and by all men in dreamless sleep. In dreamless sleep
(sushupti), the ``I'', like everything else, disappears and one
remains in one's native state -- in the true ``I'' but generally
without retaining memory of this condition on awakening.

\section*{15}

\begin{quote}
On the present the past and the future stand. They too are the present
in their times. Thus the present alone exists. Ignoring the present,
and seeking to know the past and the future, is like trying to count
without the initial unit.
\end{quote}

The present \emph{is} always, for even the past was the present in its
time, and so also will the future be the present in \emph{its}
time. Whatever happens therefore happens only in the present. When
Methuselah was born, he was born in the present, and when he died
after 9 or 10 centuries he died also in the present, despite the later
date. Similarly all that happened to him between those two events
happened also in the present. Thus the present is the only significant
tense in actuality. Moreover, let us not forget the fact that time is
made of instants which are so minute as to have no room either for a
past or for a future, but for the present alone. The next verse will
tell us that even the present is unreal, being one of the notions of
our mind, as past and future are --- acts of our memory.

\index{present}
\index{future}
\index{past}

\section*{16}

\begin{quote}
Do time and space exist apart from us? If we are the body we are
affected by time and space. But are we the body? We are the same now,
then and forever.
\end{quote}

\index{time}\index{space}\index{body}

Of course time and space are mere concepts in us. Because in our long
journey in life we pass through multitudes of experiences, we have to
conceive past, present and future in order to arrange them
conveniently in their sequence of occurence in our memory. Because we
perceive multiplicity, we have to conceive a space in which to
accomodate them, like the screen on which cinematograph pictures are
spread. Without a screen there can be no pictures. The screen on which
the universe actually appears and moves is thus our own mind, from
which it emanates as thoughts, either of external physical objects, or
of internal concepts, sensations, emotions, including the senses of
time and space. 

Those who take themselves for the body take \emph{time} to be the
creator and destroyer of all things, and thus it inspires them with
great fear --- fear of future calamaties, of death, of loss of fortune
and position, or whatever it may be. Many of them consult astrologers
to read the decrees of time and foretell events long in advance of
their occurrence. To them birth, youth, old age and death; creation,
preservation and dissolution; past, present and future; health and
disease, prosperity and adversity all exist without the shadow of a
doubt: they fall prey to time and its vagaries. The others who know
themselves to be pure spirit are bodiless, timeless and spaceless; and
Bhagavan affirms, they are thus free from the hallucination of ``We
alone are; time and space are not''.

\section*{1}

\begin{quote}

\end{quote}

\index{body}\index{sheathes}\index{world}\index{ghost}

\section*{1}

\begin{quote}

\end{quote}

\index{body}\index{sheathes}\index{world}\index{ghost}

\section*{1}

\begin{quote}

\end{quote}

\index{body}\index{sheathes}\index{world}\index{ghost}

\section*{1}

\begin{quote}

\end{quote}

\index{body}\index{sheathes}\index{world}\index{ghost}

\section*{1}

\begin{quote}

\end{quote}

\index{body}\index{sheathes}\index{world}\index{ghost}

\section*{1}

\begin{quote}

\end{quote}

\index{body}\index{sheathes}\index{world}\index{ghost}



\section*{23}

\begin{quote}
The body does not say ``I''. In sleep no one admits he is not. The
``I'' emerging, all else emerges. Inquire with a keen mind whence this
``I'' arises.
\end{quote}

The body, being insentient, knows nothing about ``I'' and ``not-I'',
yet the ``I'' persists with or without a body -- in the waking state
or in sleep or swoon - as the man who himself wakes, swoons and
sleeps. To know the true nature of this perennial ``I'', we have to
conduct an inquiry into its source.

\index{body}
\index{sleep}
\index{inquiry}
\index{I}

\section*{24}

\begin{quote}
The insentient body does not say ``I.'' The ever-existennt
consciousness is not born (thus cannot say ``I''). The ``I'' of the
size of the body springs up between the two: it is known as
chit-jada-granthi (the knot which ties together the sentient and
insentient), bondage, individuality, ego, subtle body, samsara, mind,
etc. 
\end{quote}

The body, unaware of its own existence, does not say ``I''; and the
Self which pure spirit, pure intelligence, has never come to and so,
also, does not say ``I.'' But somehow the intelligence under the
compelling power of \emph{avidya} (ignorance) assumes a body, comes to
identify itself with this body and to call itself ``I,'' thus tying
together body and soul in a knot, which is known as the knot of
ignorance in the heart -- literally the sentience-insentience knot. It
is an extremely hard knot which defies centuries of births, but breaks
of its own accord when Self-realisation is achieved, and bondage and
ignorance are destroyed forever.

\emph{``Samsara''} means going round on the wheel of birth and
death. In India, the wife is significantly also called
\emph{samsara}. 

\index{chit-jada-granthi}\index{sentient}\index{insentient}\index{body!subtle}\index{individuality}\index{bondage}\index{samsara}\index{mind}\index{knot}

\section*{25}


\begin{quote}
Know that this formless ghost (the ego or ``I'') springs up in a form
(body). Taking a form it lives, feeds and grows. Leaving a form it
picks up another, but when it is inquired into, it drops the form and
takes to flight.
\end{quote}

The ego is a vertiable ghost. A ghost is a disembodied spirit that
takes on a shadowy appearance to play the living being and hoax
people. The ego also is formless spirit -- the \emph{Atman} itself --
but it picks up a body and; without knowing it; hoaxes others as well
as itself. It begins its \emph{samsaric} career by identifying itself
with the body to enjoy the good things of the world. It reaps the
retribution of falling into abysmal \emph{avidya} (ignorance), losing
memory of its true nature, and acquiring the false notions of having a
birth, of acting, eating and growing, of accumulating wealth, marrying
and begetting children, of being diseased; hungry and miserable and
finally, of dying. But when the time of its redemption draws near, it
undertakes an investigation of its real nature, sheds its
identification with the bdoy, transcends its previous illusions and
becomes free once again, full of the bliss of self-discovery and
self-knowledge (jnana). 

\index{ego}\index{body}\index{eating}\index{growth}

\section*{26}

\begin{quote}
  The ego existing, all else exists. The ego not existing, nothing
  else exists. The ego is thus all. Inquiring as to what the ego is,
  is therefore surrendering all.
\end{quote}

Verse fourteen also makes the ego, or ``I'' the all. But here, we are
led to draw the conclusion that true surrender is the surrender of the
ego (which is the totality of the not-Self, or ``everything'') and
that the same surrender can be achieved by the method of
\emph{vichara} spoken of before.

\index{ego}\index{inquiry}

\section*{27}

\begin{quote}

  The non-emergence of the ``I'' is the state of being THAT. Without
  seeking and attaining the place whence the ``I'' emerges, how is one
  to achieve self-extinction -- the non-emergence of the ``I''?
  Without that achievement, how is one to abide as THAT -- one's true
  state? 

\end{quote}

\index{}

\emph{The non-emergence of the ``I'' means egolessness, the natural
  state of being or THAT. To stop the ego from rising we have to find
  the place of its emergence and annihilate it there, before it
  emerges, so that we may consciously ever abide as THAT, egoless, in
  the heart, as we unconsciously do in sleep. The word ``place''
  stands here for Heart.

\section*{28}

\begin{quote}
  Like the diver who dives to recover what has fallen into deep water,
  controlling speech and breath and with a keen mind, one must dive
  into himself and find whence the ``I'' emerges.
\end{quote}

The basic theme of many of the previous verses, it must have been
observed, is the \emph{vichara}, through which the search for the
ego's source has to be made. Deep diving is a metaphor that implies
salvaging the ego from the depths of ignorance into which it has
fallen, not amateurishly but very expertly and unremittingly, or else
success will be sporadic and even doubtful. Bhagavan means that this
\emph{sadhak}'s life should be dedicated to Realisation and to nothing
else, for who knows what obstacles destiny will raise against him to
bar his march to the highest in future lives? So he asks us to turn
into divers right now, contrlling speech and breath. Breath-control is
equivalent to mental silence (suspension of thoughts), whcih has to be
practised alongside the inquiry in order to train the mind to be
alone, \emph{kaivalya} (thought-free), when it will perceive itself in
its natural purity, the mot previous Self, ``whence the I emerges.''

\index{}

\section*{1}

\begin{quote}

\end{quote}

\index{body}\index{sheathes}\index{world}\index{ghost}

\section*{1}

\begin{quote}

\end{quote}

\index{body}\index{sheathes}\index{world}\index{ghost}

\section*{1}

\begin{quote}

\end{quote}

\index{body}\index{sheathes}\index{world}\index{ghost}

\section*{1}

\begin{quote}

\end{quote}

\index{body}\index{sheathes}\index{world}\index{ghost}

\section*{1}

\begin{quote}

\end{quote}

\index{body}\index{sheathes}\index{world}\index{ghost}

\section*{1}

\begin{quote}

\end{quote}

\index{body}\index{sheathes}\index{world}\index{ghost}

\section*{1}

\begin{quote}

\end{quote}

\index{body}\index{sheathes}\index{world}\index{ghost}

\section*{1}

\begin{quote}

\end{quote}

\index{body}\index{sheathes}\index{world}\index{ghost}

\section*{1}

\begin{quote}

\end{quote}

\index{body}\index{sheathes}\index{world}\index{ghost}

\section*{1}

\begin{quote}

\end{quote}

\index{body}\index{sheathes}\index{world}\index{ghost}

\section*{1}

\begin{quote}

\end{quote}

\index{body}\index{sheathes}\index{world}\index{ghost}

\section*{1}

\begin{quote}

\end{quote}

\index{body}\index{sheathes}\index{world}\index{ghost}



\chapter{Index}
\printindex

\end{document}
