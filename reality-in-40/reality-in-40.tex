\documentclass[12pt]{report}
\usepackage{hyperref}   % use for hypertext links, including those to external documents and URLs

\begin{document}

\title{Reality in 40 Verses}
\author{Ramana Maharishi}
\maketitle

\tableofcontents

\chapter{Invocation}

\section{i}
\section{ii}

\chapter{Text}

\section{15}

\begin{quote}
On the present the past and the future stand. They too are the present
in their times. Thus the present alone exists. Ignoring the present,
and seeking to know the past and the future, is like trying to count
without the initial unit.
\end{quote}

The present \emph{is} always, for even the past was the present in its
time, and so also will the future be the present in \emph{its}
time. Whatever happens therefore happens only in the present. When
Methuselah was born, he was born in the present, and when he died
after 9 or 10 centuries he died also in the present, despite the later
date. Similarly all that happened to him between those two events
happened also in the present. Thus the present is the only significant
tense in actuality. Moreover, let us not forget the fact that time is
made of instants which are so minute as to have no room either for a
past or for a future, but for the present alone. The next verse will
tell us that even the present is unreal, being one of the notions of
our mind, as past and future are --- acts of our memory.

\index{present}
\index{future}
\index{past}

\section{16}

\begin{quote}

\end{quote}


\section{23}

\begin{quote}
The body does not say ``I''. In sleep no one admits he is not. The
``I'' emerging, all else emerges. Inquire with a keen mind whence this
``I'' arises.
\end{quote}

The body, being insentient, knows nothing about ``I'' and ``not-I'',
yet the ``I'' persists with or without a body -- in the waking state
or in sleep or swoon - as the man who himself wakes, swoons and
sleeps. To know the true nature of this perennial ``I'', we have to
conduct an inquiry into its source.

\index{body}
\index{sleep}
\index{inquiry}
\index{I}


\end{document}
