\documentclass[12pt]{report}
\usepackage{hyperref}   % use for hypertext links, including those to external documents and URLs
\usepackage{makeidx} 
\makeindex

\begin{document}

\title{Reality in 40 Verses}
\author{Ramana Maharishi}
\maketitle


\tableofcontents

\chapter{Introduction}

\chapter{Invocation}

\section{i}
\section{ii}

\chapter{Text}


\section{1}

\begin{quote}
Because the world is seen, we have to infer a common (a Lord)
possessing unlimited powers to appear as the diversity. The pictures
consisting of names and forms, the seer, the canvas, the light ----
all these are He Himself.
\end{quote}

\index{world}\index{Lord}\index{diversity}\index{Self}

The Forty begins here. To understand Bhagavan's meaning we have to use
the key with which he supplies us in the Invocation. There he declares
Reality to be the thought-free Awareness which dwells in the
heart. Here he brings in the world in order to meet on their own
ground those disciples who do perceive a ``real'' external world. He
is saying something like this: ``You see a world and ascribe an
omnipotent creator to it. But as we have already seen, this creation
is only an appearance, a manifestation of that Awareness of which we
were speaking. It has no more reality in itself than have the pictures
projected on a screen.'' From the heart thoughts spontaneously rise,
like vapour from the ocean, and turn into a kaleidoscopic world of
names, forms, colours, sounds, smells and other impressions. These are
in it, or on it as on a canvas of which the heart is itself the seer
and the sight.

Pure Consciousness or Pure Mind is thus the pictures, the screen, the
seer, and the light or sight.

\section{2}

\begin{quote}
All schools of thought postulate the fundamental triad --- God, soul
and world --- although all three are manifestations of the One. The
belief that the three remain eternally lasts only as long as the ``I''
or ego lasts. To destroy the ego and remain in one's own state is best.
\end{quote}

\index{God}\index\{soul}\index{world}\index{Self}

Most religions are based on the assumption that the triad mentioned in
the text is eternal. Bhagavan rejects this assumption as being the
child of the ignorant ego which mistakes itself for the body. The
``I-am-the-body'' notion compels the admissions of an individuality
(jiva), a world, its creator, as three distinct, perennial,
co-existing entities. Bhagavan, as we have seen, perceives a single
existence of which these three are an illusory manifestation which,
however, vanishes the moment the eternal ``I'' is apprehended and the
ego perishes.

\section{3}

\begin{quote}
Of what avail to debate whether the world is real or unreal, sentient
or insentient, pleasant or unpleasant? Extingishing the ego,
transcending the world, realising the Self --- that is the state which
is dear to all, and free from the sense of unity and duality.
\end{quote}

\index{philosophy}

The same line of thought continues. Destruction of the ego is a
\emph{sine qua non} for the realisation of the Self within the
heart. It brings to an end all speculation about reality and
unreality, God and world, whose true nature will be revealed in actual
experience. This is the most blissful attainable state and beyond the
plurality of the illusory world.

\section{4}

\begin{quote}
If the Self be with form, God and the world will be also. If one be
formless oneself, how and by whom can their forms be seen? Can their
be sight without eyes? The Self is the eye, the limitless Eye.
\end{quote}

\index{form}\index{formless}

This refers to the \emph{jnani}, who although having a body seem
himself as bodiless and formless, and so cannot see God, or in fact
see anything with form. The \emph{ajnani} (the non-realised),
perceiving himself as a body, takes God also to be a body and worships
him in all sorts of material, formal representations. Yet the fact
remains that even he perceives everything through his own formless
Self, which we have granted to be the only seers, the only knowledge
there is --- the ``limitless Eye''. Those who condemn idol-worship
forget that they themselves worship material symbols and icons, and
attribute to God forms, dimensions, positions, even sentiments and
sense-perceptions exactly as they do to themselves. Having no
experience or conception of a formless omniscient spirit, they feel
literally lost lost at the idea of worshipping something not
represented in a form. God, thus, appears according to the degree of
realisation of one's Self.

``Can there be sight without eyes?'' means that without consciousness
there can be no knowledge of anything, just as without a lamp none of
the objects present in a darm room can be seen. Can there be a world
to an unconscious man?

\section{5}

\begin{quote}
The body is in the form of, and includes, the 5 sheathes. Is there a
world apart from the body? Has anyone without a body seen the world?
\end{quote}

\index{body}\index{sheathes}\index{world}\index{ghost}

The body is a complex structure containing a large number of
instruments or organs which the Self, as ego uses for a large number
of purposes, including among others those of hearing, smelling,
seeing, thinking, feeling, memorizing and reasoning. The materials out
of which these instruments or parts are made vary from the grossest to
the finest. The emph{Shastras} (scriptures) have arranged them in five
groups. To each group one sheath of \emph{kosha} is assigned. The
\emph{kosha} dealing with purely physical matter is called
\emph{annamayakosha} (the sheat of food). The \emph{pranamayakosha}
(the vital sheath) looks after the fivefold functions of the vital
energies --- breathing, assimilatin, generation, excretion and
locomotion. The \emph{manomayakosha} (mental sheath) contains the
faculties of mentation. The \emph{vijnanamayakosha} is the sheath of
the intellectual and reasoning faculties, of scientific and
philosophic thinking, and last is the \emph{anandamayakosha}, the
sheath of bliss, or causal sheath, which stores up within itslef the
karmic seeds of every birth and is concerned with that state in which
profound peace is enjoyed by the dreamless sleeper. This \emph{kosha}
is made of the finest substance, \emph{sattva}, which in itself is
happy, due to its freedom from grossness and its close proximity to
the blissful Self.

Thus the term body includes all these \emph{koshas}, whose appearance
and disappearance cause the appearance and disappearance of all
objective and subjective perceptions. Assumption of a body is
therefore necessary for the world's enjoyment and the body owes its
existence, as we shall see in the next verse, to the five senses,
which are the properties of the mind.

\section{6}

\begin{quote}
The world is but the fivefold sense-objects, which are the results of
the five senses. Since the mind perceives the world through the
senses, is there a world without the mind?
\end{quote}

\index{world}\index{senses}\index{mind}

Through the sensory organs lodges in the five \emph{koshas}, the
senses display before the mind a variety of objects --- physica,
vital, emotional, mental and intellectual. Apart from the five sense
perceptions, there all sorts of other internal senses which also arise
from the mind, work through the mind, and are understood by the mind
--- such as the senses of time, of space, or ``I'' and ``mine'', and
the artistic, ethical, religious and spiritual senses for
instance. Since all these senses form the world we known and have one 
common origin, which is the mind, the world cannot therefore be other
than that mind.

\section{1}

\begin{quote}

\end{quote}

\index{body}\index{sheathes}\index{world}\index{ghost}

\section{1}

\begin{quote}

\end{quote}

\index{body}\index{sheathes}\index{world}\index{ghost}

\section{1}

\begin{quote}

\end{quote}

\index{body}\index{sheathes}\index{world}\index{ghost}



\section{10}

\begin{quote}
Knowledge and ignorance are interrelated: the one does not exist
without the other. Inquiring to whom is it that knowledge and that
ignorance, and arriving at their root cause, the Self, this is true
knowledge. 
\end{quote}

\index{knowledge}\index{ignorance}\index{inquiry}\index{Self}

To speak of ignorance is to admit its opposite --- knowledge --- and
\emph{vice versa}. Until we become aware of an object we remain
ignorant of its existence. To learn a lesson is to admit our previous
ignorance of its content. Knowledge is thus the light which clears
away the darkness of ignorance. But knowledge and ignorance which
pertain to external objects are mere modes of thought. They come and
go, and are therefore of no consequence in the search for Truth. What
is of consequence is their knower, who is fixed, changeless, also
called first principle because he is efficient, causeless, the eternal
thinker, who precedes and survives all his thoughts --- ``the basic
One'' (verse nine).

\section{1}

\begin{quote}

\end{quote}

\index{body}\index{sheathes}\index{world}\index{ghost}

\section{1}

\begin{quote}

\end{quote}

\index{body}\index{sheathes}\index{world}\index{ghost}

\section{1}

\begin{quote}

\end{quote}

\index{body}\index{sheathes}\index{world}\index{ghost}

\section{1}

\begin{quote}

\end{quote}

\index{body}\index{sheathes}\index{world}\index{ghost}



\section{15}

\begin{quote}
On the present the past and the future stand. They too are the present
in their times. Thus the present alone exists. Ignoring the present,
and seeking to know the past and the future, is like trying to count
without the initial unit.
\end{quote}

The present \emph{is} always, for even the past was the present in its
time, and so also will the future be the present in \emph{its}
time. Whatever happens therefore happens only in the present. When
Methuselah was born, he was born in the present, and when he died
after 9 or 10 centuries he died also in the present, despite the later
date. Similarly all that happened to him between those two events
happened also in the present. Thus the present is the only significant
tense in actuality. Moreover, let us not forget the fact that time is
made of instants which are so minute as to have no room either for a
past or for a future, but for the present alone. The next verse will
tell us that even the present is unreal, being one of the notions of
our mind, as past and future are --- acts of our memory.

\index{present}
\index{future}
\index{past}

\section{16}

\begin{quote}
Do time and space exist apart from us? If we are the body we are
affected by time and space. But are we the body? We are the same now,
then and forever.
\end{quote}

\index{time}\index{space}\index{body}

Of course time and space are mere concepts in us. Because in our long
journey in life we pass through multitudes of experiences, we have to
conceive past, present and future in order to arrange them
conveniently in their sequence of occurence in our memory. Because we
perceive multiplicity, we have to conceive a space in which to
accomodate them, like the screen on which cinematograph pictures are
spread. Without a screen there can be no pictures. The screen on which
the universe actually appears and moves is thus our own mind, from
which it emanates as thoughts, either of external physical objects, or
of internal concepts, sensations, emotions, including the senses of
time and space. 

Those who take themselves for the body take \emph{time} to be the
creator and destroyer of all things, and thus it inspires them with
great fear --- fear of future calamaties, of death, of loss of fortune
and position, or whatever it may be. Many of them consult astrologers
to read the decrees of time and foretell events long in advance of
their occurrence. To them birth, youth, old age and death; creation,
preservation and dissolution; past, present and future; health and
disease, prosperity and adversity all exist without the shadow of a
doubt: they fall prey to time and its vagaries. The others who know
themselves to be pure spirit are bodiless, timeless and spaceless; and
Bhagavan affirms, they are thus free from the hallucination of ``We
alone are; time and space are not''.

\section{1}

\begin{quote}

\end{quote}

\index{body}\index{sheathes}\index{world}\index{ghost}

\section{1}

\begin{quote}

\end{quote}

\index{body}\index{sheathes}\index{world}\index{ghost}

\section{1}

\begin{quote}

\end{quote}

\index{body}\index{sheathes}\index{world}\index{ghost}

\section{1}

\begin{quote}

\end{quote}

\index{body}\index{sheathes}\index{world}\index{ghost}

\section{1}

\begin{quote}

\end{quote}

\index{body}\index{sheathes}\index{world}\index{ghost}

\section{1}

\begin{quote}

\end{quote}

\index{body}\index{sheathes}\index{world}\index{ghost}



\section{23}

\begin{quote}
The body does not say ``I''. In sleep no one admits he is not. The
``I'' emerging, all else emerges. Inquire with a keen mind whence this
``I'' arises.
\end{quote}

The body, being insentient, knows nothing about ``I'' and ``not-I'',
yet the ``I'' persists with or without a body -- in the waking state
or in sleep or swoon - as the man who himself wakes, swoons and
sleeps. To know the true nature of this perennial ``I'', we have to
conduct an inquiry into its source.

\index{body}
\index{sleep}
\index{inquiry}
\index{I}

\section{1}

\begin{quote}

\end{quote}

\index{body}\index{sheathes}\index{world}\index{ghost}

\section{1}

\begin{quote}

\end{quote}

\index{body}\index{sheathes}\index{world}\index{ghost}

\section{1}

\begin{quote}

\end{quote}

\index{body}\index{sheathes}\index{world}\index{ghost}

\section{1}

\begin{quote}

\end{quote}

\index{body}\index{sheathes}\index{world}\index{ghost}

\section{1}

\begin{quote}

\end{quote}

\index{body}\index{sheathes}\index{world}\index{ghost}

\section{1}

\begin{quote}

\end{quote}

\index{body}\index{sheathes}\index{world}\index{ghost}

\section{1}

\begin{quote}

\end{quote}

\index{body}\index{sheathes}\index{world}\index{ghost}

\section{1}

\begin{quote}

\end{quote}

\index{body}\index{sheathes}\index{world}\index{ghost}

\section{1}

\begin{quote}

\end{quote}

\index{body}\index{sheathes}\index{world}\index{ghost}

\section{1}

\begin{quote}

\end{quote}

\index{body}\index{sheathes}\index{world}\index{ghost}

\section{1}

\begin{quote}

\end{quote}

\index{body}\index{sheathes}\index{world}\index{ghost}

\section{1}

\begin{quote}

\end{quote}

\index{body}\index{sheathes}\index{world}\index{ghost}

\section{1}

\begin{quote}

\end{quote}

\index{body}\index{sheathes}\index{world}\index{ghost}

\section{1}

\begin{quote}

\end{quote}

\index{body}\index{sheathes}\index{world}\index{ghost}

\section{1}

\begin{quote}

\end{quote}

\index{body}\index{sheathes}\index{world}\index{ghost}

\section{1}

\begin{quote}

\end{quote}

\index{body}\index{sheathes}\index{world}\index{ghost}

\section{1}

\begin{quote}

\end{quote}

\index{body}\index{sheathes}\index{world}\index{ghost}



\chapter{Index}
\printindex

\end{document}
