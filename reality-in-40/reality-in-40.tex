\documentclass[12pt]{report}
\usepackage{hyperref}   % use for hypertext links, including those to external documents and URLs
\usepackage{makeidx} 
\makeindex

\begin{document}

\title{40 Verses on Reality - ``Ulladu Narpadu''}
\author{Ramana Maharishi \\ Translation and Commentary by S.S. Cohen}
\maketitle


\tableofcontents

\chapter{Preface}

It was not without much hesitation that I acceded to the suggestion of
one or two of my friends to prepare a simplified translation of Sri
Bhagavan's ULLADU NARPADU, or \emph{Forty Verses on Reality}, which
newcomers to Sri Ramanasram, especially the English-speaking
foreigners who come in increasing numbers might more easily
understand.

It is generally admitted that Sri Bhagavan's ideas are often beyond
the reach of the common reader and the beginner. They are made more
difficult by his Tamil mode of expression and by the spontaneity with
which he wrote the individual verses, for it was not his intention to
prduce a compact philosophical system or thesis. He wrote the verses
as they occurred to him, and they were later arranged by a disciple in
the order in which we see them in print.

When undertaking this venture, I placed before me six different
English translations, chose the versions common to the majority of
them, and wrote these down in an almost conversational English. I
avoided technical terms and difficult words so far as this could be
done while remaining faithful to the original. When I had a doubt due
to lack of agreement between the different translators, I sought the
help of Tamil scholars in Vellore.

I also wrote brief notes on each verse, developing the verse's main
points so that in some places the notes read like a paraphrase, but
without learned quotations or long dissertations. For all that seekers
(\emph{sadhakas}) need and want is to understand the spirit of
Bhagavan's utterances and apply it in their spiritual practice
(\emph{sadhana}).

In these forty verses, as the reader will observe, Bhagavan has
touched on all the salient points of his teaching, constantly
stressing the great value and efficacy of the \emph{vichara}, or
investigation into the nature of the investigator himself. All the
Masters of the Upanishads maintain that man is not the elements out of
which his body is made, but the mind, or intelligent principle, or
being, which uses the body. \emph{That} is the serene blissful Self,
the absolute, secondless Reality, which all are seeking consciously
and unconsciously in different ways -- devious or straight, wrong or
right -- and of which the \emph{sadhaka} endeavors to have a direct
and full Knowledge.

The synopsis which follows not only gives the gist of every verse, but
is also intended to help the reader to locate a specific subject. It
takes the place of an index, which seems out of place in a small work
like this.

S.S. Cohen

\chapter{Invocation}

\section*{i. Awareness is the nature of Reality}

\begin{quote}
Without awareness of Reality, can Reality exist? Because this
awareness-reality, itself free from thought, exists as the source of
all thoughts, it is called Heart. How to know it? To be as it is
(thought-free) in the Heart, is to know it.
\end{quote}

This verse and the next form the Invocation, which customarily
precedes spiritual and poetic works in Indian literature. It may be
addressed to a particular deity such as Ganapati, the \emph{deva} in
charge of poetic effusions, or to the \emph{devas} in general, to a
favorite \emph{devi}, or to the guru or to one of the three major
divinities. But Bhagavan, recognising a single Reality from which all
things proceed, makes his dedication to that, as the pure Awareness
(\emph{chit}) abiding in the Heart as external existence (sat), or the
absolute Brahman.

The literal translation of the first sentence of this verse reads:
'Can there be awareness of that which is other than existence?' This
makes knowledge or awareness the criterion of existence, because the
non-existent cannot make itself known. The color, for example, that is
not visible, or the sound that is not audible, amount to nothing.

\section*{ii. Fear of death is the driving force behind the quest for immortality}

\begin{quote}
  Those who have an infinite fear of death take refuge in the Feet of
  the supreme Lord Who is without birth and death. Can the thought of
  death occur to those who have destroyed their ``I'' and ``mine'' and
  have become immortal?
\end{quote}

Those who most identify themselves with the body are the people who
fear death most. Seeing the dissolution of the body they deduce their
own dissolution to be simultaneous with it, and dread the terrible
Unknown that lurks behind it. Their only hope of safety lies, there,
in the worship of the Almighty Lord, who alone is deathless.

But those who through the practice of \emph{sadhana} or spiritual
discipline have transcended this false identification no longer have
bodies to be the victims of death. Even the thought of death does not
occur to them. They are \emph{videhas}, bodiless, although they
continue to occupy a body.

This verse also implies that by taking refuge in the Lord, these
fear-torn people will, in course of time, so progress spiritually that
they will be able to destroy their sense of ``I'' and ``mine'' and
attain immortality, since the death of the ego will evidently destroy
death and the thought of death.

\chapter{Text}


\section*{1}

\begin{quote}
Because the world is seen, we have to infer a common cause (a Lord)
possessing unlimited powers to appear as the diversity. The pictures
consisting of names and forms, the seer, the canvas, the light ----
all these are He Himself.
\end{quote}

\index{world}\index{Lord}\index{diversity}\index{Self}

The Forty begins here. To understand Bhagavan's meaning we have to use
the key with which he supplies us in the Invocation. There he declares
Reality to be the thought-free Awareness which dwells in the
heart. Here he brings in the world in order to meet on their own
ground those disciples who do perceive a ``real'' external world. He
is saying something like this: ``You see a world and ascribe an
omnipotent creator to it. But as we have already seen, this creation
is only an appearance, a manifestation of that Awareness of which we
were speaking. It has no more reality in itself than have the pictures
projected on a screen.'' From the heart thoughts spontaneously rise,
like vapour from the ocean, and turn into a kaleidoscopic world of
names, forms, colours, sounds, smells and other impressions. These are
in it, or on it as on a canvas of which the heart is itself the seer
and the sight.

Pure Consciousness or Pure Mind is thus the pictures, the screen, the
seer, and the light or sight.

\section*{2}

\begin{quote}
All schools of thought postulate the fundamental triad --- God, soul
and world --- although all three are manifestations of the One. The
belief that the three remain eternally lasts only as long as the ``I''
or ego lasts. To destroy the ego and remain in one's own state is best.
\end{quote}

\index{God}\index{soul}\index{world}\index{Self}

Most religions are based on the assumption that the triad mentioned in
the text is eternal. Bhagavan rejects this assumption as being the
child of the ignorant ego which mistakes itself for the body. The
``I-am-the-body'' notion compels the admissions of an individuality
(jiva), a world, its creator, as three distinct, perennial,
co-existing entities. Bhagavan, as we have seen, perceives a single
existence of which these three are an illusory manifestation which,
however, vanishes the moment the eternal ``I'' is apprehended and the
ego perishes.

\section*{3}

\begin{quote}
Of what avail to debate whether the world is real or unreal, sentient
or insentient, pleasant or unpleasant? Extingishing the ego,
transcending the world, realising the Self --- that is the state which
is dear to all, and free from the sense of unity and duality.
\end{quote}

\index{philosophy}

The same line of thought continues. Destruction of the ego is a
\emph{sine qua non} for the realisation of the Self within the
heart. It brings to an end all speculation about reality and
unreality, God and world, whose true nature will be revealed in actual
experience. This is the most blissful attainable state and beyond the
plurality of the illusory world.

\section*{4}

\begin{quote}
If the Self be with form, God and the world will be also. If one be
formless oneself, how and by whom can their forms be seen? Can their
be sight without eyes? The Self is the eye, the limitless Eye.
\end{quote}

\index{form}\index{formless}

This refers to the \emph{jnani}, who although having a body seem
himself as bodiless and formless, and so cannot see God, or in fact
see anything with form. The \emph{ajnani} (the non-realised),
perceiving himself as a body, takes God also to be a body and worships
him in all sorts of material, formal representations. Yet the fact
remains that even he perceives everything through his own formless
Self, which we have granted to be the only seers, the only knowledge
there is --- the ``limitless Eye''. Those who condemn idol-worship
forget that they themselves worship material symbols and icons, and
attribute to God forms, dimensions, positions, even sentiments and
sense-perceptions exactly as they do to themselves. Having no
experience or conception of a formless omniscient spirit, they feel
literally lost lost at the idea of worshipping something not
represented in a form. God, thus, appears according to the degree of
realisation of one's Self.

``Can there be sight without eyes?'' means that without consciousness
there can be no knowledge of anything, just as without a lamp none of
the objects present in a darm room can be seen. Can there be a world
to an unconscious man?

\section*{5}

\begin{quote}
The body is in the form of, and includes, the 5 sheathes. Is there a
world apart from the body? Has anyone without a body seen the world?
\end{quote}

\index{body}\index{sheathes}\index{world}\index{ghost}

The body is a complex structure containing a large number of
instruments or organs which the Self, as ego uses for a large number
of purposes, including among others those of hearing, smelling,
seeing, thinking, feeling, memorizing and reasoning. The materials out
of which these instruments or parts are made vary from the grossest to
the finest. The emph{Shastras} (scriptures) have arranged them in five
groups. To each group one sheath of \emph{kosha} is assigned. The
\emph{kosha} dealing with purely physical matter is called
\emph{annamayakosha} (the sheat of food). The \emph{pranamayakosha}
(the vital sheath) looks after the fivefold functions of the vital
energies --- breathing, assimilatin, generation, excretion and
locomotion. The \emph{manomayakosha} (mental sheath) contains the
faculties of mentation. The \emph{vijnanamayakosha} is the sheath of
the intellectual and reasoning faculties, of scientific and
philosophic thinking, and last is the \emph{anandamayakosha}, the
sheath of bliss, or causal sheath, which stores up within itslef the
karmic seeds of every birth and is concerned with that state in which
profound peace is enjoyed by the dreamless sleeper. This \emph{kosha}
is made of the finest substance, \emph{sattva}, which in itself is
happy, due to its freedom from grossness and its close proximity to
the blissful Self.

Thus the term body includes all these \emph{koshas}, whose appearance
and disappearance cause the appearance and disappearance of all
objective and subjective perceptions. Assumption of a body is
therefore necessary for the world's enjoyment and the body owes its
existence, as we shall see in the next verse, to the five senses,
which are the properties of the mind.

\section*{6}

\begin{quote}
The world is but the fivefold sense-objects, which are the results of
the five senses. Since the mind perceives the world through the
senses, is there a world without the mind?
\end{quote}

\index{world}\index{senses}\index{mind}

Through the sensory organs lodges in the five \emph{koshas}, the
senses display before the mind a variety of objects --- physical,
vital, emotional, mental and intellectual. Apart from the five sense
perceptions, there all sorts of other internal senses which also arise
from the mind, work through the mind, and are understood by the mind
--- such as the senses of time, of space, or ``I'' and ``mine'', and
the artistic, ethical, religious and spiritual senses for
instance. Since all these senses form the world we known and have one 
common origin, which is the mind, the world cannot therefore be other
than that mind.

\section*{7}

\begin{quote}
Although the world and the awareness of it rise and set together, it
is by awareness that the world is known. The source from which they
both rise, and into which they both set, always shines without itself
rising or setting. That alone is real.
\end{quote}

\index{world}\index{awareness}

This verse is reminiscent of the Invocation and confirms the previous
verse, which make awareness the criterion of existence as well as the
source of the world. Awareness ``always shines'' as the ``limitless
Eye'' mentioned in verse four, the eternal Knower. It goes without
saying that the appearance of the world is simultaneous with the
awareness of it, and disappearance of the world simultaneous with the
withdrawal of that awareness. For the fact of the awareness of the
world is the fact of its existence. We cannot affirm the existence of
an object without first affirming awareness of it. Therefore awareness
is the only Reality there is.

\section*{8}

\begin{quote}
In whatever name and form the nameless and formless is worshipped,
therein lies the path of its realisation. Realising one's truth as the
truth of that reality, and merging into it, is true realisation.
\end{quote}

\index{form}

All roads lead to Rome --- all sincere worship comes from the heart,
and leads to the formless God in the heart. To believe that one's
reality is the same as God's is an important step towards the
realisation of Him as Pure Consciousness and the process of merging
into Him. How many millions of innocent human beings would have been
spared the horror of religious persecutions throughout the centuries
in the name of God, and how many wars would have been prevented, had
this truth been accepted as the one truth underlying all religions,
the basic world faith!

\section*{9}

\begin{quote}
The dyads and triads rest on the basic One. Inquiring about that One
in the mind, they will disappear. Those who see thus are the seers of
truth: they remain unruffled.
\end{quote}

\index{dyad}\index{triad}

The dyads are the pairs of opposites --- knowledge and ignorance,
light and darkness, happiness and miserty, birth and death, etc. The
triad is triple principle of seen, seer and sight; object, subject and
the perception of the former by the latter. As all the numbers stand
on, and originate from, the first number, so are the dyads and triads
based on, arising from, and ofthe same nature as the one seer, the
perceiving mind. He who realises the world as such retains a uniform
serenity in all conditions of life.


\section*{10}

\begin{quote}
Knowledge and ignorance are interrelated: the one does not exist
without the other. Inquiring to whom is it that knowledge and that
ignorance, and arriving at their root cause, the Self, this is true
knowledge. 
\end{quote}

\index{knowledge}\index{ignorance}\index{inquiry}\index{Self}

To speak of ignorance is to admit its opposite --- knowledge --- and
\emph{vice versa}. Until we become aware of an object we remain
ignorant of its existence. To learn a lesson is to admit our previous
ignorance of its content. Knowledge is thus the light which clears
away the darkness of ignorance. But knowledge and ignorance which
pertain to external objects are mere modes of thought. They come and
go, and are therefore of no consequence in the search for Truth. What
is of consequence is their knower, who is fixed, changeless, also
called first principle because he is efficient, causeless, the eternal
thinker, who precedes and survives all his thoughts --- ``the basic
One'' (verse nine).

\section*{11}

\begin{quote}
Is it not ignorance to know all but the all-knowing Self? When the
latter, the substratum of both knowledge and ignorance, is known,
knowledge and ignorance themselves both disappear.
\end{quote}

\index{knowledge}\index{ignorance}

It is of course foolish to know about everything in the world, and
remain ignorant of one's own Self. Knowledge of the perishable -- the
universe and its contents -- perishes with the body, and cannot be
transferred to another body, except perhaps as tendencies or abilities
in the perishable too, which may not have any spiritual value in a
future life. The imperishable alone endures and gives imperishable
satisfaction, and this lies wholly within ourselves, who are the
source and ground of both knowledge and ignorance -- that is, of all
experiences whatever.

\section*{12}

\begin{quote}
True knowledge is neither knowledge or ignorance. Objective knowledge
is not true knowledge. Because the Self is self-effulgent, having no
second to know or be known, it is Supreme Knowledge -- not empty
nothingness.
\end{quote}

\index{knowledge!objective}\index{void}

This continues the theme of verses ten and eleven. We have seen tht
objective knowledge is knowledge of the perishable, the apparent, the
non-existent, the unreal (see Invocation). Self-awareness is true
knowledge, because it is absolute, i.e. changeless, non-dual,
ever-pure (thought-free). This purity is not emptiness because of the
lack of perceivable objectives in it, but the ever-shining plenum of
Awareness-Being (\emph{chit-Sat}).

\section*{13}

\begin{quote}
The Self alone is knowledge, is truth. Knowledge of the diversity is
ignorance, is false knowledge. Yet ignorance is not apart from the
Self, which is knowledge. Are the ornaments different from the gold
which is real?
\end{quote}

\index{diversity}

So the world with all its multiplicity of shapes, colors, smells
tastes and so forth is nothing but pure consciousness in substance,
like variously-shaped jewelry which is nothing but gold. To perceive
shapes, colors, smells and the like as different from one another is
ignorance, is illusion, but to see them as the single substance out of
which they are made -- the pure mind -- is true knowledge.

``Yet ignorance is not apart from the Self'' because all experiences
as thoughts come from the Self and are witnessed by it (verses six and
seven). 

\section*{14}

\begin{quote}
The ``I'' existing, ``you'' and ``he'' also exist. If by investigating
the truth of the ``I'' the ``I'' ceases, ``you'' and ``he'' will also
cease and will shine as the One. This is the natural state of one's
being.

\end{quote}

``You'' and ``he'' are the world; it stands and falls with the ``I''
or ego, which constructs it. Realising one's being is realising the
whole world to be the same effulgent being -- ``the One''. This state
of being is experienced by the Self-realised man in the waking state
consciously and by all men in dreamless sleep. In dreamless sleep
(sushupti), the ``I'', like everything else, disappears and one
remains in one's native state -- in the true ``I'' but generally
without retaining memory of this condition on awakening.

\section*{15}

\begin{quote}
On the present the past and the future stand. They too are the present
in their times. Thus the present alone exists. Ignoring the present,
and seeking to know the past and the future, is like trying to count
without the initial unit.
\end{quote}

The present \emph{is} always, for even the past was the present in its
time, and so also will the future be the present in \emph{its}
time. Whatever happens therefore happens only in the present. When
Methuselah was born, he was born in the present, and when he died
after 9 or 10 centuries he died also in the present, despite the later
date. Similarly all that happened to him between those two events
happened also in the present. Thus the present is the only significant
tense in actuality. Moreover, let us not forget the fact that time is
made of instants which are so minute as to have no room either for a
past or for a future, but for the present alone. The next verse will
tell us that even the present is unreal, being one of the notions of
our mind, as past and future are --- acts of our memory.

\index{present}
\index{future}
\index{past}

\section*{16}

\begin{quote}
Do time and space exist apart from us? If we are the body we are
affected by time and space. But are we the body? We are the same now,
then and forever.
\end{quote}

\index{time}\index{space}\index{body}

Of course time and space are mere concepts in us. Because in our long
journey in life we pass through multitudes of experiences, we have to
conceive past, present and future in order to arrange them
conveniently in their sequence of occurence in our memory. Because we
perceive multiplicity, we have to conceive a space in which to
accomodate them, like the screen on which cinematograph pictures are
spread. Without a screen there can be no pictures. The screen on which
the universe actually appears and moves is thus our own mind, from
which it emanates as thoughts, either of external physical objects, or
of internal concepts, sensations, emotions, including the senses of
time and space. 

Those who take themselves for the body take \emph{time} to be the
creator and destroyer of all things, and thus it inspires them with
great fear --- fear of future calamaties, of death, of loss of fortune
and position, or whatever it may be. Many of them consult astrologers
to read the decrees of time and foretell events long in advance of
their occurrence. To them birth, youth, old age and death; creation,
preservation and dissolution; past, present and future; health and
disease, prosperity and adversity all exist without the shadow of a
doubt: they fall prey to time and its vagaries. The others who know
themselves to be pure spirit are bodiless, timeless and spaceless; and
Bhagavan affirms, they are thus free from the hallucination of ``We
alone are; time and space are not''.

\section*{1}

\begin{quote}

\end{quote}

\index{body}\index{sheathes}\index{world}\index{ghost}

\section*{1}

\begin{quote}

\end{quote}

\index{body}\index{sheathes}\index{world}\index{ghost}

\section*{1}

\begin{quote}

\end{quote}

\index{body}\index{sheathes}\index{world}\index{ghost}

\section*{1}

\begin{quote}

\end{quote}

\index{body}\index{sheathes}\index{world}\index{ghost}

\section*{1}

\begin{quote}

\end{quote}

\index{body}\index{sheathes}\index{world}\index{ghost}

\section*{1}

\begin{quote}

\end{quote}

\index{body}\index{sheathes}\index{world}\index{ghost}



\section*{23}

\begin{quote}
The body does not say ``I''. In sleep no one admits he is not. The
``I'' emerging, all else emerges. Inquire with a keen mind whence this
``I'' arises.
\end{quote}

The body, being insentient, knows nothing about ``I'' and ``not-I'',
yet the ``I'' persists with or without a body -- in the waking state
or in sleep or swoon - as the man who himself wakes, swoons and
sleeps. To know the true nature of this perennial ``I'', we have to
conduct an inquiry into its source.

\index{body}
\index{sleep}
\index{inquiry}
\index{I}

\section*{24}

\begin{quote}
The insentient body does not say ``I.'' The ever-existennt
consciousness is not born (thus cannot say ``I''). The ``I'' of the
size of the body springs up between the two: it is known as
chit-jada-granthi (the knot which ties together the sentient and
insentient), bondage, individuality, ego, subtle body, samsara, mind,
etc. 
\end{quote}

The body, unaware of its own existence, does not say ``I''; and the
Self which pure spirit, pure intelligence, has never come to and so,
also, does not say ``I.'' But somehow the intelligence under the
compelling power of \emph{avidya} (ignorance) assumes a body, comes to
identify itself with this body and to call itself ``I,'' thus tying
together body and soul in a knot, which is known as the knot of
ignorance in the heart -- literally the sentience-insentience knot. It
is an extremely hard knot which defies centuries of births, but breaks
of its own accord when Self-realisation is achieved, and bondage and
ignorance are destroyed forever.

\emph{``Samsara''} means going round on the wheel of birth and
death. In India, the wife is significantly also called
\emph{samsara}. 

\index{chit-jada-granthi}\index{sentient}\index{insentient}\index{body!subtle}\index{individuality}\index{bondage}\index{samsara}\index{mind}\index{knot}

\section*{25}


\begin{quote}
Know that this formless ghost (the ego or ``I'') springs up in a form
(body). Taking a form it lives, feeds and grows. Leaving a form it
picks up another, but when it is inquired into, it drops the form and
takes to flight.
\end{quote}

The ego is a vertiable ghost. A ghost is a disembodied spirit that
takes on a shadowy appearance to play the living being and hoax
people. The ego also is formless spirit -- the \emph{Atman} itself --
but it picks up a body and; without knowing it; hoaxes others as well
as itself. It begins its \emph{samsaric} career by identifying itself
with the body to enjoy the good things of the world. It reaps the
retribution of falling into abysmal \emph{avidya} (ignorance), losing
memory of its true nature, and acquiring the false notions of having a
birth, of acting, eating and growing, of accumulating wealth, marrying
and begetting children, of being diseased; hungry and miserable and
finally, of dying. But when the time of its redemption draws near, it
undertakes an investigation of its real nature, sheds its
identification with the bdoy, transcends its previous illusions and
becomes free once again, full of the bliss of self-discovery and
self-knowledge (jnana). 

\index{ego}\index{body}\index{eating}\index{growth}

\section*{26}

\begin{quote}
  The ego existing, all else exists. The ego not existing, nothing
  else exists. The ego is thus all. Inquiring as to what the ego is,
  is therefore surrendering all.
\end{quote}

Verse fourteen also makes the ego, or ``I'' the all. But here, we are
led to draw the conclusion that true surrender is the surrender of the
ego (which is the totality of the not-Self, or ``everything'') and
that the same surrender can be achieved by the method of
\emph{vichara} spoken of before.

\index{ego}\index{inquiry}

\section*{27}

\begin{quote}

  The non-emergence of the ``I'' is the state of being THAT. Without
  seeking and attaining the place whence the ``I'' emerges, how is one
  to achieve self-extinction -- the non-emergence of the ``I''?
  Without that achievement, how is one to abide as THAT -- one's true
  state? 

\end{quote}

\index{}

\emph{The non-emergence of the ``I'' means egolessness, the natural
  state of being or THAT. To stop the ego from rising we have to find
  the place of its emergence and annihilate it there, before it
  emerges, so that we may consciously ever abide as THAT, egoless, in
  the heart, as we unconsciously do in sleep. The word ``place''
  stands here for Heart.

\section*{28}

\begin{quote}
  Like the diver who dives to recover what has fallen into deep water,
  controlling speech and breath and with a keen mind, one must dive
  into himself and find whence the ``I'' emerges.
\end{quote}

The basic theme of many of the previous verses, it must have been
observed, is the \emph{vichara}, through which the search for the
ego's source has to be made. Deep diving is a metaphor that implies
salvaging the ego from the depths of ignorance into which it has
fallen, not amateurishly but very expertly and unremittingly, or else
success will be sporadic and even doubtful. Bhagavan means that this
\emph{sadhak}'s life should be dedicated to Realisation and to nothing
else, for who knows what obstacles destiny will raise against him to
bar his march to the highest in future lives? So he asks us to turn
into divers right now, contrlling speech and breath. Breath-control is
equivalent to mental silence (suspension of thoughts), whcih has to be
practised alongside the inquiry in order to train the mind to be
alone, \emph{kaivalya} (thought-free), when it will perceive itself in
its natural purity, the mot previous Self, ``whence the I emerges.''

\index{}

\section*{29}

\begin{quote}
  Seeking the source of the ``I'' with a mind turned inwards and no
  uttering of the word ``I'' is indeed the path of
  knowledge. Meditation on ``I am not this, I am that'' is an aid to
  the inquiry, but not the inquiry itself.
\end{quote}

\index{inquiry}

Bhagavan misses no opportunity of reminding us that the quest ``who am
I?'' is not a formula to be repeated mechanically like an incantation,
but an intellectual activity into the nature of the ``I'' which is
carried out until its base is fully grasped and its source is
reached. The whole process is dialectical, involving the exercise of
the logical faculty, till it ends in the silence of the heart, which
transcends all faculties. Some suggestive formula such as ``I am
THAT'' may be used to being with, but in course of time it has to turn
into an unshakeable conviction, side by side with the stilling of the
mind as mentioned in the previous commentary, which gradually grows in
depth and duration. That is why the path of the \emph{vichara} is
known as the path of knowlege (\emph{jnana marga}).

\section*{30}

\begin{quote}
  Inquiring ``Who am I?'' within the mind, and reaching the heart, the
  ``I'' collapses. Instantly, the real ``I'' appears (as ``I'' ``I''),
  which although it manifests itself as ``I'' is not the ego, but the
  true being.
\end{quote}

\index{heart}

What happens to the ``I'' which has found its own source and
collapsed? The meaning is that the ``I'' which has not been aware of
its own reality has now, through inquiry, come face to face with it,
and has turned from the notion of being a mortal body to the
realisation of being the shining sea of consciousness. This is the
'collapse' of the false ``I'' giving place to the true ``I'', which is
eternally present as ``I'', ``I'', ``I'' without end or beginning. We
must not forget that there is only one, secondless ``I'', whether we
view it as ego, totally sunk in the pleasures of the world and in
ignorance, or as Self, the substratum and source of the world.

``Inquring within the mind who am I?'' is an affirmation once again,
that the quest has to be carried out with the mind.

\section*{31}

\begin{quote}
  What remains to be done by him, who, having extinguished the ego,
  remains immersed in the bliss of the Self? He is aware of nothing
  but the Self. Who can understand his state?
\end{quote}

\index{aware}\index{Self}

The purpose of all human endeavors, conscious or unconscious, is the
gaining of happiness. The unwise seeks it outside himself in wealth,
matrimony, high political and social positions, fame, worldy
achievements and pleasure of all sorts. The wise knows that the
happiness that comes from an outside cause is illusory due to its
precarious nature and its inability even temporarily to confer
contentment without trouble, fear, and endless anxiety. Lasting,
undiluted happiness is one's very nature, and thus within the grasp of
anyone who earnestly seeks it. One who has gained this inner beatitude
has no further actions to do, nor purpose to achieve. All his
aspirations having been fulfilled, his sole preoccupation remains that
ocean of bliss, which passes the understanding of the common man.

\section*{32}

\begin{quote}
  Despite the Vedas proclaiming ``Thou art THAT'', it is sheer
  weak-mindedness not to investigate into the nature of oneself and
  abide as the Self, but instead to go on thinking ``THAT I am, not
  this.'' 
\end{quote}

\index{vedas}

The main point of this verse is that when the Vedas tell us that we
are THAT, we are in duty bound to conduct an inquiry into ourselves in
order to experience \emph{the truth of it} and abide at THAT or the
Self, rather than just mechanically thinking that we are not the body
but THAT. Investigation and meditation will eventually rise above the
body-thought, and will reach the \emph{tanumanai} state (the rarified
mind) through which the pure awareness can be directly
apprehended. This is the silent heart itself.

\section*{33}

\begin{quote}
  It is ludicrous to thik ``I know myself'', or ``I do not know
  myself'', admitting thereby two selves, one the object of the
  other. That the Self is only one is the experience of all.
\end{quote}

\index{self-knowledge}

To know a thing is to create a duality -- the knower and the
known. But in self-knowledge there can be no duality, the known being
the knower himself, the object and the subject being one and the same
identify.

It is common experience that the ``I'' is unqualified and single: it
is neither divisible into parts nor tainted by qualities. However fat
or lean, old or young, learned or ignorant, rich or poor, whole or
dismembered one may be, one is aware of oneself only as ``I'' devoid
of any attributes. The bare ``I'', ``I'', ``I'' is the primary
cognition of everyone, preceding the ``mine'' cognition, the body and
all its appertenances, and all its thoughts. This shows that the Self
is non-dual, homogeneous and indivisble, and can abide pure by itself
with no thoughts to disturb it, being itself not a thought, but the
intuitive recognition of oneself as the eternal knower, the pivot --
more correctly, the substance -- of all one knows. It is evident that
the ``I'' being pure indivisible consciousness, is experienced by the
\emph{jnani} as the same in all.

\section*{34}

\begin{quote}
  Without trying to realise in the heart that reality which is the
  true nature of all, and without trying to abide in it, to engage in
  disputations as to ther the reality exists or not, or is real or
  not, denotes delusion born of ignorance.
\end{quote}

\index{philosophy}

The theme of the previous verse continues. The realization of one's
Self is the realization of the true nature of all else, the Self being
single and homogeneous. Disputations deepen the ignorance and not
infrequently lead to acrimony, anger, hatred, and jealousy among the
disputants, not to speak of the vanity and arrogance they create in
the hearts of the winners. They should thus be shunned by seekers of
Truth and of Peace everlasting.

\section*{35}

\begin{quote}
  To seek and abide in that which is always attained is true
  attainment. All other attainments, such as siddhis (thaumaturgic
  powers), are like those acquired in dreams, which prove to be unreal
  on waking. Can they who are established in reality and are rid of
  illusions be ensnared by them?
\end{quote}

\index{dream}\index{siddhi}\index{waking}

Sometimes we dream that we are flying in the air, or leaping over
precipices hundreds of feet wide, or stopping a running motor car with
a light touch of the hand, or doing things which, in the waking state,
would appear miraculous, yet prove unreal on waking. The
\emph{siddhis} exhibited in the waking state appear to the man who has
freed himself from illusion exactly like the dream miracles ---
utterly false. The greatest of all miracles and all \emph{siddhis} is
the discovery of and eternal abidence in, oneself.

In the olden days occasionally a \emph{siddhi}-mad youth used to come
to Ramanashram with the intention of using Bhagavan's presence to
promote the success of his pursuit of \emph{siddhis}. One or two of
them were reasonable enough to listen to the advice of the devotees
and quit the Ashram before it was too late. \footnote{Cohen said
  ``... quit the Ashram betimes'', but because I am favoring American
  English over British, I translated it.} But one, more persistent
that the others, continued to interfere with his uvula and the
posterior membrane of his tongue, ignoring all advice to desist, until
after two or three weeks his people had to be called to take him
away. These were lucky to be saved from the pitfalls of
\emph{siddhis}. Many others had their \emph{siddhis} turned on them
like boomerangs adversely affecting their physical and mental
constitutions. \emph{Siddhis} come naturally to the very few, due to
yogic practices carried out during their previous \emph{sadhana} and
\emph{karmic} determinations. These people are safe and sometimes
helpful to humanity, if the behave reasonably in the \emph{sadhana} of
this life. They are likely to attain \emph{mukti} if they are lucky
and favorably disposed --- \emph{sattvic} in other words.

``That which is always attained'' refers to the Self, which is always
present at the true nature of the ego whether ego is conscious of it
or not (see comment on verse 30) before the birth of the body, during
its existence and after its disintegration at death.

\section*{36}

\begin{quote}
  The thought ``I am not the body'' helps on to meditate ``I am not
  this: I am THAT'' and to abide as THAT. But why should one forever
  think ``I am THAT''? Does a man need to always think ``I am a man''?
  We are always THAT.
\end{quote}

\index{}

Verse thirty-two discourages the use of the thought ``II am not
this''. However, this verse avers that even this negative meditation
is useful to the extend that it leads to the positive meditation ``I
am THAT''. But even the latter meditation appears to the jnani
superfluous, in that it is already granted that one is always THAT ---
``That which is always attained'' (verse thirty-five). That we are not
the body any thinking man can discover for himself without even
attempts at Self-realisation. For what dullard can find no difference
between himself and, say, a chair or table which does not move, think
of speak like him, yet is made ofthe same elements? There must
certainly be something in the human body, over and above what there is
in the other objects. That something is life, or mind, or knowledge,
or THAT, which \emph{sadhakas} try to isolate from the body and
perceive by itself, in its aloneness (\emph{kaivalya}). \emph{That} is
the Self-realisation or self-cognition we are after.

\section*{37}

\begin{quote}
  The theory that in practical life duality prevals, whereas
  non-duality prevails in the (spiritual) attainment, is
  false. Whether one is still anxiously searching for the Self, or has
  actually attained it, one is not other than the tenth man.
\end{quote}

\index{duality}

Non-duality always prevails, whether viewed from the viewpoint of the
world or from that or the realised yogi. The realization of Self
cannot turn the dual into the non-dual. The truth of non-duality
stands eternally true, as verse one has shown.

The \emph{tenth man} refers to the story in which ten men travelled
together. After fording a river, they decided to count themselves to
make sure that none of them had been lost in the crossing. The man who
counted his nine companions forgot to count himself, which resulted in
their starting a search for the tenth man --- actually always present
as the counter himself. The same applies to man, who is always present
as the eternal non-dual reality, but imagines himself always in
duality due to his perception of multiplicity --- ``I'' and ``you'',
the chair, the door, the window and a million other objects. But the
realized man is free from this false imagination: he knows himself to
be the tenth man.

\section*{38}

\begin{quote}
  So long as a man feels himself the doer, he reaps the fruits of his
  actions. But as soon as he realizes through inquiry who is the doer,
  the sense or doership drops off and the threefold karma comes to an
  end. This is the final Liberation.
\end{quote}

\index{karma}\index{doer}

Who is the doer? If the body is the doer then we have to attribute
intelligence to it, an intelligence which it does not possess. The
identification of the instrument of an act with the actor is the cause
of much trouble. An illustration will be to the point. A man has a
grudge against another man and plans to do away with him. He waylays
him on a dark night, takes a stone and kills him with it. Who is the
killer? Certainly not the stone,  although it is the stone that has
done the evil deed, nor is the hand which holds the stone, nor the
body of which the hand is a part, and which is as insentient, and
there as innocent, as the stone. It is the mind which, with hatred,
planned and executed the crime, using the instrumentality of the body
and the stone. Therefore the mind is the empirical man, or ego who, so
long as he believes himself to be the actor, has to reap, the fruit of
his actions effected through a body. But this belief, like the ego
itself, is not permanent: it passes away immediately an inquiry is
made into the identity of the doer.

The triple karma which hangs around the neck of the doer is made up of
the \emph{sanchita} (accumulated karma), the \emph{prarabdha} (the
karma which is destined to be worked out in this birth), and the
\emph{agami} (the karma which becomes active in future births). The
last class of karma will remain unfulfilled in the case of the person
who has attained Liberation in the present body, and who will have no
other births for karma to be worked in.

Questions are sometimes asked concerning the \emph{jnani's prarabdha}
as to why it does not cease with his attainment of \emph{jnana}, thus
sparing him suffering that may arise in the form of virulent disease,
with which some famous \emph{jnanis} are known to be burdened. The
answer is that \emph{prarabdha} of the \emph{jnani} had been allotted
to him at or before birth, when he was still liable to the working of
karma prior to his attainment of \emph{jnana}. As for his suffering,
it is not as paniful to him as it appears to others: it is greatly
mitigated by the Realization which unceasingly wells up in his heart.

Some Biblically-oriented Westerners seem to think that the suffering
of the \emph{jnani} is due to his taking upon himself the sins of his
disciples. Vedanta denies to transfer of sins and its
responsibilities. Strict justice is the law of karma which tolerates
no one to suffer for another's crimes, least of all a Guru, who comes
to show the way to Truth. Far from being punished he is rewarded by
the service, love and devotion of the disciples. Thus the belief in a
Salvation through the vicarious suffering of the Master is totally
unacceptable in this path, where each man is regarded as working out
his own liberation through hard word, self-purification, worship of
the Guru, self-control, spiritual practices and a full sense of moral
responsibility. In the whole Vedantic literature one does not find a
single reference to the transference of sins, but always to karma.


\section*{39}

\begin{quote}
  Bondage and Liberation exist so long as thoughts of bondage and
  liberation exist. These come to an end when an inquiry is made into
  the nature of he who is bound or free, and the ever-present and
  ever-free Self is realized.
\end{quote}

\index{bondage}\index{liberation}

This has a close resemblance to the last verse, which makes the sense
of doership to be the cause of karma. Likewise the sense of being
bound or free makes bondage and liberation exist. Thus wrong notions
about oneself are responsible for all the acts of destiny: birth,
death, bondage, ignorance, etc. But wrong notions can be rectified by
right knowledge, which can be had only throuh and inquiry into the
nature of the person who is the victim of wrong notions. Then his real
Self will reveal itself and will dispel all notions, all senses, and
all thoughts, including the sense and thought of \emph{jivahood}
(individuality) itself.

\section*{40}

\begin{quote}
  It is said that Liberation is with form or without form, or with and
  without form. Let me tell you that Liberation destroys all three as
  well as the ego which distinguishes between them.
\end{quote}

\index{liberation}\index{philosophy}

All these forms of liberation, some of which are said to take place in
a disembodied state in some supersensuous worlds --- \emph{Vaikuntha},
\emph{Satyaloka}, etc. are hypothetical. At best they offer
encouragement to the \emph{sadhakas} who are partial to them. The fact
of the matter is that true and absolute Liberation results only from
\emph{jnana} (knowledge of the Absolute), which alone can destroy
ignorance, either in this body or in one of the following bodies. For
there are no planes or states of consciousness where radical salvation
is possible, other than the waking state, i.e., in a body, where
bondage and ignorance are felt and attempts for redemption made; least
of all in the state of after-death where there is no body to feel the
limitations and retributions of karma.

Therefore he who aspires to reach the highest has to exert himself
hard here and now, preferably by the \emph{vichara} method which
Bhagavan has so graciously propounded and so often reiterated in these
verses. The determined \emph{sadhaka} will not fail to verify these
truths by his own experience if he puts them to the test, full of
confidence, in his own self and the unfailing silent support of the
Master, who is not other than the very Reality he is so earnestly
seeking, and who ever and ever abides in his own heart as Existence,
Consciousness ande Bliss --- \emph{Sat Chit Ananda}.

OM SHANTI SHANTI SHANTIHI



\chapter{Index}
\printindex

\chapter{Postscript}

I, Terrence Monroe Brannon, decided that there needed to be a freely
available version of the one book which summarizes Advaita Vedanta. I
first learned of this text in my dealings with the guru Arunachala
Ramana and his spiritual institute AHAM.com

I took the liberty of changing British English to American English.

Feedback/corrections are welcome: TheQuietCenter\@gmail.com or
http://www.LivingCosmos.org

\chapter{Back Cover}

A twentieth century Master of Advaita discourses on the discovery of
that Absolute Reality from which all things proceeed, and which is
also a person's own real Self. The translation and commentaries are by
one who knew him and lived close to him for many years.

\end{document}
